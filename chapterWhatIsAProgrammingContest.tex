\section{Voorrondes}
Voor studenten van de Vrije Universiteit Amsterdam kent de International Collegiate Programming Contest (ICPC) in totaal vier rondes:
\begin{enumerate}
\item Amsterdam Algorithm Programming Preliminaries (AAPP): Open voor alle teams van de Vrije Universiteit Amsterdam die aan de voorwaarden voldoen, zie ook Appendix \ref{EligibilityDecisionTree}. De drie beste teams mogen naar BAPC.
\item Benelux Algorithm Programming Contest (BAPC): De beste drie teams van elke onderwijsinstelling doen mee aan de BAPC. De twee beste teams mogen naar NWERC, mits het team uit drie personen bestaat (zie ook Appendix \ref{EligibilityDecisionTree}).
\item Northwestern Europe Regional Contest (NWERC): De beste twee teams van elke onderwijsinstelling doen mee aan de NWERC. De drie beste teams mogen naar de ICPC World Finales.
\item International Collegiate Programming Contest (ICPC) World Finales: De beste drie teams van de hele regio Noord-West Europa mogen meedoen aan de ICPC World Finals.
\end{enumerate}
Teams die niet voldoen aan de Eligibility Decision Tree (zie appendix \ref{EligibilityDecisionTree}), worden meestal toegelaten als spectator indien daarvoor plek is.

\section{Algemene opzet wedstrijd}
Elk team (van maximaal drie studenten) probeert binnen vijf uur zoveel mogelijk problems (meestal tussen de $10$ en de $13$ opgavens) op te lossen door het maken van programma's op \'e\'en computer. Het programma leest gegevens uit een invoerbestand, zoekt of berekent het juiste antwoord, en schrijft de resultaten als uitvoer.

De problems zijn meestal gebaseerd op bekende klassieke algoritmen, zoals het kortste pad algoritme van Dijkstra, of backtracking. Vaak komen er ook een aantal wiskundige opgaven in voor.

Overleggen mag logischerwijs alleen met hun teamgenoten. Verder is het toegestaan om een cheatsheet (Team Reference Document) te gebruiken en om een eigen toetsenbord mee te nemen. Teams die een problem goed hebben binnen vier uur, krijgen een balloon van een balloonbabe in de kleur van de vraag.

\section{Submissions insturen}
Submissions kan men inzenden via de juryinterface DomJudge (een website). Verdere toegang tot het internet is afgeschermd voor de teams. Via de juryinterface krijgt men te horen of de inzending goed is. Het programma wordt beoordeeld op de volgende twee criteria: correctheid en effici\"entie. Dat wil zeggen: het programma moet een vooraf bepaalde set testcases (dat kan ook een case zijn die niet gegeven is als input in de problems) correct oplossen binnen een vooraf vastgestelde tijdslimiet (meestal enkele seconden).

De mogelijke reacties van de jury zijn onder andere:
\begin{itemize}
\item Accepted;
\item Wrong Answer;
\item Timelimit Exceeded;
\item Runtime Error.
\end{itemize}
Voor elke goede oplossing (Accepted) krijgt men een ballon van een balloon babe.

\section{Programmeertalen}
De onderstaande talen zijn toegestaan bij de genoemde contests:
\begin{itemize}
\item[AAPP] Java, C, C++, C++11, C\#, Haskell
\item[BAPC] Java, C, C++
\item[NWERC] Java, C,  C++
\item[ICPC World Finales] Java, C, C++
\end{itemize}

\section{Puntentelling}
De score van een team bestaat uit twee onderdelen:
\begin{itemize}
\item Het aantal opgaven opgelost in vijf uur
\item De totale tijd (de penalty time)
\end{itemize}
Per opgeloste opgave bestaat dat uit de volgende twee onderdelen:
\begin{itemize}
\item Het aantal minuten tussen het begin van de wedstrijd en het oplossen van de opgave
\item $20$ strafminuten voor elke foute inzending
\end{itemize}
Een foute inzending levert alleen strafminuten op als de opgave later alsnog wordt opgelost.

De teams worden gesorteerd op aantal opgeloste opgaven. Teams met evenveel opgaven worden gesorteerd op tijd. Het laatste uur wordt het scorebord bevroren (freeze). Je hoort dan nog wel of je eigen inzendingen goed of fout zijn, maar niet welke opgaven de andere teams nog oplossen.

Zie voor een uitgebreider uitleg het reglement, Appendix \ref{Rulebook} (Judgement).

\section{Tijdschema}
Elke programmeerwedstrijd heeft dezelfde opzet qua tijdschema. Sommige kiezen er echter voor om deze indeling uit te rekken over een weekend, om zo meer ruimte te cre\"eren voor excursies en sponsoring praatjes. Dit is meestal het geval bij NWERC, omdat deelnemende teams ook uit het buitenland komen. Door teams wordt het uitrekken van de planning vooral als vervelend ervaren, omdat het evenement veel langer duurt dan nodig is.

Het tijdschema ziet er als volgt uit:
\begin{itemize}
\item[Registratie] Het registreren van teamsleden. Organisatie deelt de verplichte gesponserde T-shirts en geeft de eventuele goodiebags. Eventueel controleren en innemen van cheatsheets. Toetsenborden kunnen eventueel ook gecontroleerd worden ($30$ minuten).
\item[Welkomswoord] Teams worden verwelkomt door de voorzitter en de hoofdsponsor. Uitleg van de spelregels, het doornemen van de dagplanning en een praatje van de hoofdsponsor (maximaal $30$ minuten).
\item[Testsessie] Voor de contest is er een testsessie. Deze testsessie wordt gebruikt om te kijken of de wedstrijdomgeving aan alle verwachtingen voldoet (maximaal $1$ uur).
\item[Coach meeting] Meeting voor de coach, zie ook sectie \ref{CoachMeeting} (maximaal $30$ minuten).
\item[Lunch] Meestal verzorgd door de hoofdsponsor ($1$ uur).
\item[Last remarks] Laatste uitleg, vragen of teams nog brandende vragen hebben (hooguit een kwartiertje, maar er wordt $30$ minuten gerekend zodat iedereen tijd heeft om op zijn plek te gaan zitten voor de contest).
\item[Contest] Contest begint ($5$ uur).
\item[Freeze scoreboard] Scoreboard staat op freeze, men kan alleen hun eigen inzendingen zien en of deze goed zijn of niet ($4$ uur na dat de contest is begonnen).
\item[Borrel] Borrelen ($1$ uur)
\item[Prijsuitreiking] Uiteindelijke scoreboard wordt gepresenteerd ($10$ minuten)
\item[Presentatie problems] Presentatie met de oplossingen wordt gepresenteerd door de jury ($15$ minuten).
\item[Dinner] Optioneel: hangt er vanaf of de commissie geld heeft om het eten te vergoeden voor de teams.
\end{itemize}