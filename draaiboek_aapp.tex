\documentclass[11pt]{report}
\usepackage{titlesec}
\titleformat{\chapter}
{\filcenter\normalfont\Large\bfseries}
{\chaptertitlename~\thechapter} {0.5em} {}

\usepackage[english]{babel}
\usepackage{graphicx}
\usepackage{amsmath,amssymb}
\usepackage{bbm}
\usepackage{listings} %listing R code
\usepackage{siunitx} %voor 10^
\usepackage[usenames,dvipsnames]{color}
\usepackage{color}
\usepackage{enumitem}
\usepackage{pdfpages}
\usepackage{titling}
\usepackage{hyperref}

\usepackage{amssymb}

\usepackage[T1]{fontenc}
\usepackage[utf8]{inputenc}
\usepackage{charter}
\usepackage{environ}
\usepackage{tikz}
\usetikzlibrary{calc,matrix}

%%%%%%%%%%%%%%%%%%%%%%%%%%%%%%%%
% code by Andrew:
% http://tex.stackexchange.com/a/28452/13304
\makeatletter
\let\matamp=&
\catcode`\&=13
\makeatletter
\def&{\iftikz@is@matrix
  \pgfmatrixnextcell
  \else
  \matamp
  \fi}
\makeatother

\newcounter{lines}
\def\endlr{\stepcounter{lines}\\}

\newcounter{vtml}
\setcounter{vtml}{0}

\newif\ifvtimelinetitle
\newif\ifvtimebottomline
\tikzset{description/.style={
  column 2/.append style={#1}
 },
 timeline color/.store in=\vtmlcolor,
 timeline color=red!80!black,
 timeline color st/.style={fill=\vtmlcolor,draw=\vtmlcolor},
 use timeline header/.is if=vtimelinetitle,
 use timeline header=false,
 add bottom line/.is if=vtimebottomline,
 add bottom line=false,
 timeline title/.store in=\vtimelinetitle,
 timeline title={},
 line offset/.store in=\lineoffset,
 line offset=4pt,
}

\NewEnviron{vtimeline}[1][]{%
\setcounter{lines}{1}%
\stepcounter{vtml}%
\begin{tikzpicture}[column 1/.style={anchor=east},
 column 2/.style={anchor=west},
 text depth=0pt,text height=1ex,
 row sep=1ex,
 column sep=1em,
 #1
]
\matrix(vtimeline\thevtml)[matrix of nodes]{\BODY};
\pgfmathtruncatemacro\endmtx{\thelines-1}
\path[timeline color st] 
($(vtimeline\thevtml-1-1.north east)!0.5!(vtimeline\thevtml-1-2.north west)$)--
($(vtimeline\thevtml-\endmtx-1.south east)!0.5!(vtimeline\thevtml-\endmtx-2.south west)$);
\foreach \x in {1,...,\endmtx}{
 \node[circle,timeline color st, inner sep=0.15pt, draw=white, thick] 
 (vtimeline\thevtml-c-\x) at 
 ($(vtimeline\thevtml-\x-1.east)!0.5!(vtimeline\thevtml-\x-2.west)$){};
 \draw[timeline color st](vtimeline\thevtml-c-\x.west)--++(-3pt,0);
 }
 \ifvtimelinetitle%
  \draw[timeline color st]([yshift=\lineoffset]vtimeline\thevtml.north west)--
  ([yshift=\lineoffset]vtimeline\thevtml.north east);
  \node[anchor=west,yshift=16pt,font=\large]
   at (vtimeline\thevtml-1-1.north west) 
   {\textsc{Timeline \thevtml}: \textit{\vtimelinetitle}};
 \else%
  \relax%
 \fi%
 \ifvtimebottomline%
   \draw[timeline color st]([yshift=-\lineoffset]vtimeline\thevtml.south west)--
  ([yshift=-\lineoffset]vtimeline\thevtml.south east);
 \else%
   \relax%
 \fi%
\end{tikzpicture}
}
%%%%%%%%%%%%%%%%%%%%%%%%%%%%%%%%

\newcommand{\half}{\frac{1}{2}}
\newcommand{\pref}[1]{(\ref{#1})}
\newcommand{\itab}[1]{\hspace{0em}\rlap{#1}}
\newcommand{\tab}[1]{\hspace{.4525\textwidth}\rlap{#1}}

\pretitle{%
  \begin{center}
  \LARGE
  \includegraphics[width=11.35cm]{./Attachments/logoAAPP16}\\[\bigskipamount]
}
\posttitle{\end{center}}

\title{Management Manual for Programming Contests}
\author{Mylène Martodihardjo}
\date{\today}

%%%%%%%%%%%%%%%%%%%%%%%%%%%%%%%%%%

\begin{document}
\selectlanguage{english}
\maketitle
\tableofcontents
\clearpage

\chapter{Introduction}
Thank you for trying to read this gigantic pile of text about organising a programming contest. This management manual is specific meant for the committee of studyassociation STORM, but can be used for any other programming contest.

\chapter{Definitions}
\begin{description}
\item[AAPP:]
The Amsterdam Algorithm Programming Preliminaries, also referred as the contest, is a programming contest for VU students. The contest is organised by studyassociation STORM. It takes place mid/end september. AAPP are the preliminaries of BAPC.

\item[BAPC:]
The Benelux Algorithm Programming Contest, organised by a studyassociation in the Benelux. It takes place in the end of october. Since $2016$, BAPC are the preliminaries of NWERC instead of AAPP.

\item[NWERC:]
The Northwestern Europe Regional Contest. It takes place in the end of november. NWERC are the preliminaries of ICPC World Finales.

\item[ICPC World Finales]
The International Collegiate Programming Contest (ICPC) World Finales are held around march.

\item[Organisation:]
The members of the organising committee of STORM, also called AAPPCie.

\item[Website:]
Wordt onderhouden door de organisatie en bevat onder andere informatie, problems van vorige jaren en de regels van de AAPP. De website is beschikbaar op \url{http://www.storm.vu/aapp}.

\item[Jury:]
Groep mensen die verantwoordlijk zijn voor het controleren van de antwoorden op de submission van deelnemers.

\item[Tech:]
Groep mensen die verantwoordlijk zijn voor het systeem.

\item[Balloon girls:]
Runners die verantwoordlijk zijn voor het uitdelen van printjes, het beantwoorden van vragen en het uitdelen van ballonen aan teams die een submission correct hebben ingeleverd.

\item[Crew:]
Organisatie, leden van de jury, tech en balloon girls. 

\item[Deelnemers:]
Leden van een deelnemende team die meedoen aan de contest.

\item[Submission:]
De submission van een oplossing door een team, welke ingeleverd kan worden via DomJudge  en die gecontroleerd zal worden door onze nakijkservers.
\end{description}

\chapter{How does a programming contest works?}
\section{Preliminaries}
For students of the VU Amsterdam, the International Collegiate Programming Contest (ICPC) knows four rounds:
\begin{enumerate}
\item Amsterdam Algorithm Programming Preliminaries (AAPP): Open for all VU Amsterdam teams which satifsy the conditions, see also the Appendix \ref{EligibilityDecisionTree}. The top three teams are allowed to BAPC.
\item Benelux Algorithm Programming Contest (BAPC): The top three teams of each educational institution can accompete to BAPC. The top two teams of  each educational institution are allowed to NWERC, provided that the team consists three team members (see also Appendix \ref{EligibilityDecisionTree}).
\item Northwestern Europe Regional Contest (NWERC): The top two teams of each educational institution can accompete to NWERC. The top three teams of  each educational institution are allowed to the ICPC World Finales.
\item International Collegiate Programming Contest (ICPC) World Finales: The top three teams of the region Northwestern Europe are allowed to the ICPC World Finals.
\end{enumerate}
Teams that do not satisfy the Eligibility Decision Tree (see appendix \ref{EligibilityDecisionTree}), are usually allowed as spectator if there is room for them.

\section{General Concept of the Contest}
Each team (with a maximum of three students) is trying to solve as many problems (most of the time between the $10$ and the $13$) as possible in five hours, by programming a program on the computer. The program reads the input from the input file, search or computes the right answer and gives back the results as output.

Most of the time, the problems are based on known classic algorithms, like the shortest path algorithm of Dijkstra or backtracking. Often there are a number of mathematical problems.

Logically, may only discuss with their teammates. Furthermore, it is allowed to use a cheat sheet (Team Reference Document) and to bring their own keyboard. Teams that have a problem correctly within four hours, gets a balloon of a balloon babe in the color of the problem.

\section{Sending in a Submissions}
Via the jury interface DomJudge (a website), teams can hand in the submissions. Further access to the internet has been blocked during the contest. The jury interface also tells the teams, if the submissins was correct or not. The program will be rated on the following two criteria: accuracy and efficiency.

That means: the program has solve a set of test cases (not only the one given as input in the problem) in previously set time limit (most of the time a few secondes).

The following reactions of the jury are possible:
\begin{itemize}
\item Accepted;
\item Wrong Answer;
\item Timelimit Exceeded;
\item Runtime Error.
\end{itemize}
For each correct solution (Accepted), you will get a balloon from a balloon babe (if the correct solution was hand in within the first four hours).

\section{Program languages}
The following program languages are accepted at the contest:
\begin{itemize}
\item[AAPP] Java, C, C++, C++11, C\#, Haskell
\item[BAPC] Java, C, C++
\item[NWERC] Java, C,  C++
\item[ICPC World Finales] Java, C, C++
\end{itemize}

\section{Score}
The team score depends on two parts:
\begin{itemize}
\item The number of correct solved problems within the five hours
\item The total time (including the penalty time).
\end{itemize}
%
There are two parts per solved problem:
\begin{itemize}
\item The numbers of minutes since the start of the contest and the moment till solving the problem %Het aantal minuten tussen het begin van de wedstrijd en het oplossen van de opgave
\item $20$ penalty time for each wrong solution %strafminuten voor elke foute inzending
\end{itemize}
A wrong solution only gives penalty time if the problem is solved later the contest.

On the scoreboard, teams will be sorted by the numbers of solved problems. Teams who has same amount of solved problems, will be sorted on total time. In the last hour of the contest, the scoreboard will be freezed. You will see from your own team if a problem is correct or not, but it is not possible to see on the scoreboard if others teams have hand in a (correct) solution.

See for more information the rulebook, Appendix \ref{Rulebook} (Judgement).

\section{Time Schedule}
Each program contest has the same set up when its come to time. Some organisations decided to spread this event over a weekend, to create some room for excursions and sponsoring activities. Most of the time, this is the case at NWERC, since teams are coming from far. However, teams likely do not like it that it takes that long before they can really start with the contest.

The time schedule for the contest is as follow:
\begin{itemize}
\item[Registration] Registration of the incoming teams. The organisation gives teams and coaches goodiebags and sponsored T-shirts, which we are obligatory to wear during the day(s). At the registation, it is also possible that the organisation wants to check the cheatsheets and the keyboards ($30$ minuten).
\item[Welkomswoord] Teams worden verwelkomt door de voorzitter en de hoofdsponsor. Uitleg van de spelregels, het doornemen van de dagplanning en een praatje van de hoofdsponsor (maximaal $30$ minuten).
\item[Testsessie] Voor de contest is er een testsessie. Deze testsessie wordt gebruikt om te kijken of de wedstrijdomgeving aan alle verwachtingen voldoet (maximaal $1$ uur).
\item[Coach meeting] Meeting voor de coach, zie ook sectie \ref{CoachMeeting} (maximaal $30$ minuten).
\item[Lunch] Meestal verzorgd door de hoofdsponsor ($1$ uur).
\item[Last remarks] Laatste uitleg, vragen of teams nog brandende vragen hebben (hooguit een kwartiertje, maar er wordt $30$ minuten gerekend zodat iedereen tijd heeft om op zijn plek te gaan zitten voor de contest).
\item[Contest] Contest begint ($5$ uur).
\item[Freeze scoreboard] Scoreboard staat op freeze, men kan alleen hun eigen inzendingen zien en of deze goed zijn of niet ($4$ uur na dat de contest is begonnen).
\item[Borrel] Borrelen ($1$ uur)
\item[Prijsuitreiking] Uiteindelijke scoreboard wordt gepresenteerd ($10$ minuten)
\item[Presentatie problems] Presentatie met de oplossingen wordt gepresenteerd door de jury ($15$ minuten).
\item[Dinner] Optioneel: hangt er vanaf of de commissie geld heeft om het eten te vergoeden voor de teams.
\end{itemize}

\chapter{Committee Tasks}
\label{Functieverdeling}
\section{Chairman}
	Leads the commission and ensures that everything runs smoothly. He delegates and motivates the members of the commission to do there tasks (see also section \ref{motivated}). He also leads the meetings (see also section \ref{leadingMeetings}). Carries responsibility for the committee. \label{responsibilityChairman} Read also the responsibility of a contest director: \ref{responsibilityChairman}.
	
	\subsection{Leading the Meetings}\label{leadingMeetings}
	Plan regular meetings, but only do so when its \underline{necessary} because nobody likes pointless meetings. Make sure everyone gets the chance to bring something up during the meeting.
	
	Design for the agenda is as follow (Dutch/English):
	\begin{enumerate}
	\item Opening
	\item Agenda
	\item Notulen Vorige Vergadering (NVV) / Minutes of Previous Meeting
	\item Post, Email en Mededelingen (PEM) / Post, Receive Email and Announcements
	\item Oude Actiepunten / Old Action Points
	\item \textit{\{kies een onderwerp\}} / \textit{\{choose a subject\}}
	\item \textit{\{kies een onderwerp\}} / \textit{\{choose a subject\}}
	\item Wat Verder Ter Tafel Komt (WVTTK) / Any Other Business (AOB)
	\item Samenwerkingsrondje (censuur) / Informal Round (censorship)
	\item Datum Volgende Vergadering (DVV) / Date Next Meeting
	\item Sluiting / Closing
	\end{enumerate}
	
	And here is a timeline for planning a meeting:\newline
	
	\begin{vtimeline}[timeline color=green!80!blue,description={text width=11cm}, 
	row sep=2ex, 
	use timeline header,
	timeline title={Plan a Meeting 101}]
	Day 0 & Think if you have time to have a meeting \endlr	
	Day 1 & Send scheduler + action points \endlr
	Day 5 & Reminder: fill in the scheduler \endlr
	Day 6 & Read the minutes \endlr
	Day 6 & Set up the agenda \endlr
	Day 7 & Mail: date of the meeting + minutes + agenda + action points \endlr
	Day 7 & Reserveve: conference room \endlr
	Day 8 - 14 & Print: minutes + agenda \endlr
	Day 8 - 14 & Reminder: meeting \endlr
	Day 8 - 14 & Meeting! \endlr
	Sometime & Wait for the rough copy of the minutes \endlr
	Sometime & Do your action points! \endlr
	\end{vtimeline}
	
	\subsubsection{Motivated}\label{motivated}
	Keeping members motivated can ensure that they would like to continue to help the committee. In order to ensure that they remain motivated, there must be a clear goal. Not only the main goal (to organise a beautiful contest for our participants), but also the action points should be as clear as possible otherwise these are not carried out.
	
	Furthermore, its helps reminding members to do there action points and to ask (friendly) if they have finished there action points. It also helps to ask if they need any help from other members to carry out the task.
	
	Solidarity with the committee also ensures that a member gets more sense to carry out their work. Eating together before/after the meeting, pie during the meeting and after the meeting working together at a task in the Stuka works.
	
	\subsection{Presentations and Speeches}
	Somebody needs to talk at
	\begin{itemize}
	\item the opening ceremony (see also \ref{openingCeremony});
	\item the last remarks (see also \ref{lastRemarks});
	\item the award ceremony (see also \ref{awardCeremony});
	\end{itemize}
	and that person is you! Prepare on time your presentations (save as PDF) and speeches.
	
	\subsection{Tips}
	\begin{itemize}
	\item Start on time with organising the contest!
	\item Keep in touch with important people:
		\begin{itemize}
		\item Board members;
		\item Chairmen of other BAPC committee in the Benelux;
		\item Coaches in the Benelux.
		\end{itemize}
	\item Do every week something for your committee.
	\end{itemize}
	
\section{Secretary}
	\subsection{Maintain the Mailbox}
	Label incoming mail and processing:
	\begin{enumerate}
	\item Forward to the right person(s)
	\item Answering
	\item Archive 
	\item Delete	
	\end{enumerate}	
	
\section{Treasurer}
	\subsection{Keep Track of Finances}
	
	\subsection{Budget}
	
	\subsection{Financial Report}
	
	\subsection{Funding}
	
\section{Assessor}
	
\section{Contest Director}
Charman of the contest, decides together with the judges (see also \ref{judge}) and the tech (see also \ref{tech}) if the contest can start and makes decisions about the contest. Carries responsibility for the contest. \label{responsibilityContestDirector} Read also the responsibility of a chairman: \ref{responsibilityChairman}.

\section{Sponsoring\label{sponsoringTasks}}
Commissar extern takes care of sponsoring. Sponsoring is needed to compensate the costs, to have a beautiful location and to give better prizes to the winners. See also \ref{sponsoring} for more inside information.
	
\section{Tech}\label{techTasks}
The tech is taking care of the technical part of organising a programming contest, see also \ref{tech} for more inside information.

\section{Judge}\label{judge}

\section{Balloon babes}
Als teams een opdracht goed hebben binnen vier uur, krijgen ze een ballon van een balloon babe. Deze schoonheid zal deze ballonen netjes vastzetten aan het bureau of aan een bureaustoel. Naast deze zware taak, doen ze ook het volgende voor de wedstrijd:
\begin{itemize}
\item Helpen bij de registratiebalie
\item Controleren dat niemand begint, voordat de contest is begonnen
\item Controleren dat niemand de opgave opent, voordat de contest is begonnen
\item Mobieltjes laten inleveren.
\end{itemize}
{\color{white}test}\\
%%%
En tijdens de wedstrijd voeren ze de volgende taken uit:
\begin{itemize}
\item Ballonnen uitdelen
\item Printjes aangeven
\item Meelopen (roken, wc, eten)
\item Controleren dat men niet vals speelt.
\end{itemize}

\section{Coaches}
Coach zijn van een team bij BAPC of NWERC, houdt in dat je het team ondersteunt daar waar nodig is en om de commissie te vertegenwoordigen bij eventuele coachmeetings 

	\subsection{Voorbereiding}
	Ondersteuning geven aan een team houdt in dat je het onderstaande zal moeten regelen of voorbereiden:
	\begin{itemize}
	\item Vervoer naar de wedstrijd (auto, OV etcetera)
	\item De registratie, zodat men kan deelnemen aan de competitie. Sowieso invoeren in ICPC, maar soms zijn er ook aanvullende formulieren (the local registration form).
	\item Koffie vinden
	\item Aanmelden bij de registratie balie. Vergeet dan niet de onderstaande informatie bij de hand te hebben:
		\begin{itemize}
		\item 	Teamnamen
		\item 	Totaal aantal personen inclusief coach(es), voor het aantal goodiebags
		\item 	Aantal mannen, aantal vrouwen inclusief coach(es), voor het geval dat er wel vrouwen T-shirts zijn.
		\item 	T-shirtmaten (voor vrouwen geldt dat als het unisex/mannen T-shirts zijn, een maat kleiner nemen dan normaal gesproken wordt gekozen)
		\end{itemize}
	\item Cheatsheets uitprinten/regelen
	\item Training regelen of zorgen dat ze zelf trainen
	\item Het team eraan herinneren dat je je eigen toetsenbord mee mag nemen
	\end{itemize}
	\textbf{Tip: het kan handig zijn om een Whatsapp groep te starten met de teamleden en de coaches erin.}

	Nadat de wedstrijd is begonnen, mag de coach zich niet meer op de contest ground bevinden. \textbf{Tip: neem wat mee om de vijf uur door te komen.}
	
	\subsection{Coach meeting}\label{CoachMeeting}
	General meeting:
	\begin{itemize}
	\item Current contest update
	\item Next year contest location (update of the bid $2016$-$2017$, geen slaapplekken in Bath. Goedkoopste optie is YHA Bath. Maximaal $120$ teams, zie ook http://staff.bath.ac.uk/masjhd/NWERC/.)
	\item Future location. Temp. bid. Prio voor landen waar NWERC nog niet is geweest.
	\end{itemize}		
	
	Meestal wordt er tijdens de test sessie of tijdens de wedstrijd, een coach meeting gehouden om eventuele problemen over de wedstrijd zelf te kunnen bespreken. Omdat je de commissie/universiteit vertegenwoordigd, is het handig om met de commissie de onderstaande vragen te bespreken voor vertrek. De volgende vragen kunnen daarnaast nog besproken worden:
	\begin{itemize}
	\item Welke universiteit organiseert de volgende editie(s) van de wedstrijd?
	\item Hoe zijn de voorrondes gegaan?
	\end{itemize}	


\chapter{Things You Need to Do if You Want a Programming Contest}
\section{Algemeen}
Hieronder vindt je alle taken die moeten worden uitgevoerd, die niet te maken hebben met sponsoring of het technische deel van de wedstrijd.		
	\subsection{Registratie starten}
	
	\subsection{Promotie}
		\subsubsection{Maandelijks mailing}
	
		\subsubsection{Posters}
	
		\subsubsection{Onderwijsco\"ordinatoren laten mailen naar studie maillijst}

	\subsection{Contact teams}
		\subsubsection{Deelnemers informeren}
		Week voor de contest begint
	
		\subsubsection{Deelnemers informatie verwerken}
		ICPC \& Google Drive
	
		\subsubsection{Evaluatie}
		
		\subsubsection{Data sturen}
		
	\subsection{Kleding}
		\subsubsection{Crew}
		
		\subsubsection{Deelnemers}
		
	\subsection{VU Diensten}
		\subsubsection{IT}
		STORM en IT hebben goede banden met elkaar. Hierdoor is het mogelijk om spullen te lenen voor de contest, mits we deze netjes terugbrengen:
		\begin{itemize}
		\item Computers\label{pcs}\\
		Vraag rond juni aan IT S\&E Werkplek Ondersteuning om $25$ pc's vrij te houden voor de "rekenwedstrijd". Waarschijnlijk heeft Willem hier al eerder aan gedacht dan wij.
		\item Printer(s)
		\item Kooikar(ren)
		\item Monitorkabels
		\item Kluis
		\item Ricoh papier sleutel
		\end{itemize}
		
		\subsubsection{FCO}
		VU busje
			
	\subsection{Bedankjes}
		\subsubsection{IT}
		Willem houdt van slagroomtaart en Emilio houdt van alles waar chocolade in zit of wat met liefde is gemaakt.
		
		\subsubsection{Jury}
		Spelletjes

\section{Sponsoring}
	\subsection{Locatie (hoofdsponsor}
	
	\subsection{Prijzen}
	
	\subsection{Goodiebags}
	
	\subsection{Bedrijventeams}
	
	\subsection{Overige reclame}
	
\section{Tech}
	\subsection{Cli\"ent}
	
	\subsection{DOMJudge}
	
	\subsection{DOMjura}
	
	\subsection{Printen}
	
	\subsection{Data opslaan}
	
	\subsection{Puppet}
	
\section{Inpaklijst}
Op \'e\'en of andere manier vergeten we op de dag zelf altijd wat.. Hier moet dus een inpaklijst komen te staan. Of ergens in de Appendix zodat je alleen de losse pagina hoeft uit te printen.

\chapter{One week before the contest}
	\section{Dag voor de contest}
		\subsection{Printen}
		\begin{itemize}
		\item[$\square$] De opgaven
		\item[$\square$] Het regelement
		\item[$\square$] Excelsheet met teamnamen, deelnemers, maten en bijzonderheden (toetsenbord, cheatsheet)
		%\item[$\boxtimes$] A closed item.
		\end{itemize}

\chapter{Idee\"en}
\begin{itemize}
\item Volgende keer alle compilers testenop de server en clients!
\item In de inschrijf form apart voor- en achternaam.
\item Duidelijkheid over alcoholgebruik.
\item Zelfde shirt als via?
\item Proberen de opgaven/oplossingen/testdata eerder te krijgen van landelijke BAPC commissie.
\item Bedankjes voor de jury op de dag van de contest.
\item - DOMjudge in de vakantie maken
\item Draaiboeken schrijven - technisch kant
\item Image bewaren
\item Eerstejaars programmeer wedstrijd
\item ICPC - adres gegevens zijn nodig
\item T-shirts zijn fijner dan polo's
\item Duidelijker formulier - volledig namen graag
\item Geen Python meer
\item Ondersteunen qua talen alleen wat NWERC ondersteund
\item Extra scherm inpakken
\item Extra stekkerdozen inpakken
\item Pas als de jury akkoord geeft, dan pas balloon geven. Balloon interface geeft eerder een balloon dan dat de jury correct geeft.
\item DOMjura
\item 2x printers
\item Balloon babes inlichten over taken
\item Cheatsheet controleren
\item Prijzen: steam tegoed. Rasberry pi
\item Training geven (Madelon, Renske)
\end{itemize}

\chapter{Appendix}
%Zie de volgende pagina's voor:
%\begin{itemize}
%\item Commissie samenstelling door de jaren
%\item Winnaars voorrondes
%\item Eligibility Decision Tree
%\item Domjudge Team Manual including a NWERC scoreboard example
%\end{itemize}
%\clearpage

\section{Wall of Fame: Crew}
	\subsection{Organisation 2014-2015}
		\subsubsection{Committee}
		De commissie bestond uit de volgende mensen:
	
	\subsection{Organisation 2013-2014}
		\subsubsection{Committee}				
		Het samenwerkingsverband UvA-VU leverde de volgende samenstelling op:
		\begin{itemize}
		\item Renske Augustijn (Voorzitter, VU)
		\item Myl\`ene Martodihardjo (Vice-voorzitter, VU)	
		\item Bram van den Akker (Sponsoring, UvA)
		\item Daniel Maaskant (Sponsoring, VU)
		\item Michael Vasseur (System Director, VU)
		\item Tirza Jochemsen (VU)
		\item Ruben Helsloot (VU)
		\item Iris Meerman (UvA)
		\item Bas van den Heuvel (UvA)
		\end{itemize}		
		
		\subsubsection{Balloon babes}
		\begin{itemize}
		\item Myl\`ene Martodihardjo (Chief balloon babe)
		\item Sherida van den Bent (VU)
		\item Nicolette Stassen (VU)
		\item Saskia Kreuzen (VU)
		\item Ysbrand Galama (UvA)
		\end{itemize}
		
		\subsubsection{Jury}
		\begin{itemize}
		\item Frank Blom (Head judge)
		\item Alex ten Brink (Jurymember of BAPC 2014 finale)
		\end{itemize}		

	\subsection{Organisation 2012-2013}
		\subsubsection{Committee}		
		There was not really a committee that year, but the following members did help to set up the preliminaries:
		\begin{itemize}
		\item Myl\`ene Martodihardjo (algemeen, fotograaf)
		\item Michael Vasseur (tech, fotograaf)
		\item Mark Laagland (tech)
		\item Jip de Beer (sponsoring)
		\item Kylie van de Moot (fotograaf)
		\end{itemize}
		
		\subsubsection{Balloon babes}
		\begin{itemize}
		\item Myl\`ene Martodihardjo
		\item Kylie van de Moot
		\end{itemize}
		
		\subsubsection{Jury}
		\begin{itemize}
		\item Michael Vasseur
		\item Mark Laagland
		\end{itemize}

\section{Winnaars voorrondes}

\section{Eligibility Decision Tree}\label{EligibilityDecisionTree}

%\addcontentsline{toc}{section}{Eligibility Decision Tree}\label{EligibilityDecisionTree}
\includepdf[pages=-1]{./Attachments/EligibilityDecisionTree-2016.pdf}

\begin{center}
\vspace*{\fill}
This page is intentionally left blank.
\vspace*{\fill}
\end{center}
\clearpage

\section{DOMjudge team manual}\label{DJteam}
%\addcontentsline{toc}{section}{DOMjudge team manual}
\includepdf[pages=-]{./Attachments/DOMjudge/team-manual.pdf}

\section{Amsterdam Algorithm Programming Preliminaries}\label{Rulebook}
\includepdf[pages=-]{./Attachments/Rules/AAPP_Rules_2016.pdf}
%\includepdf[pages=1-3]{./Attachments/Rules/BAPC_Rules_2015.pdf} %pages=- (om alles in te laden van de PDF) geeft errors

\end{document}