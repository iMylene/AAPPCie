\section{Contest}
\begin{description}
\item[Participants:]
Members of a team, who are joining the contest. A team has a limit of three participants.

\item[Submission:]
The submission for a problem, which can be hand in via DomJudge during the contest and will be check by our own servers.

\item[Vrije Universiteit Amsterdam:] Also referred as VU Amsterdam, is a university in Amsterdam (the Netherlands), founded in $1880$ by Abraham Kuyper. The literal translation of the Dutch name Vrije Universiteit is "Free University", where "Free" refers to independence of the university from both the State and Church.

\item[AAPP:]
The Amsterdam Algorithm Programming Preliminaries is a programming contest for VU students. The contest is organised by the VU in collabration with studyassociation STORM. It takes place mid/end September. The AAPP are the preliminaries of the BAPC.

\item[BAPC:]
The Benelux Algorithm Programming Contest, organised by an institute (in collabration with a studyassociation) in the Benelux. It takes place near the end of October. Since $2016$, BAPC are the preliminaries of NWERC for VU teams instead of AAPP.

\item[NWERC:]
The Northwestern Europe Regional Contest. It takes place in the end of November. NWERC are the preliminaries of ICPC World Finales.

\item[ICPC World Finales]
The International Collegiate Programming Contest (ICPC) World Finales are held around march.
\end{description}

\section{Crew}
\begin{description}
\item[Crew:]
Organisation, members of the jury, tech and balloon girls.

\item[Organisation:]
The members of the organising committee of STORM, also called AAPPCie.

\item[Jury:]
Members of the organisation, who are responsible to check the submission of the contestants.

\item[Tech:]
Members of the organisation, who are responsible for the system.

\item[Balloon girls:]
Runners who are responsible for handing out balloons after a submission is correct, for handing out prints and for answering your questions during the contest.

\item[Website:]
Will be maintained by the organisation and contains information about problems from last year and the rules of AAPP. The website is availible at \url{http://www.storm.vu/aapp}.
\end{description}