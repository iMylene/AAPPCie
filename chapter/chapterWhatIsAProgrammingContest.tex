\section{Preliminaries}
For students of the VU Amsterdam, the International Collegiate Programming Contest (ICPC) knows four rounds:
\begin{enumerate}
\item Amsterdam Algorithm Programming Preliminaries (AAPP): Open for all VU Amsterdam teams which satifsy the conditions, see also the Appendix \ref{EligibilityDecisionTree}. The top three teams are allowed to BAPC.
\item Benelux Algorithm Programming Contest (BAPC): The top three teams of each educational institution can accompete to BAPC. The top two teams of  each educational institution are allowed to NWERC, provided that the team consists three team members (see also Appendix \ref{EligibilityDecisionTree}).
\item Northwestern Europe Regional Contest (NWERC): The top two teams of each educational institution can accompete to NWERC. The top three teams of  each educational institution are allowed to the ICPC World Finales.
\item International Collegiate Programming Contest (ICPC) World Finales: The top three teams of the region Northwestern Europe are allowed to the ICPC World Finals.
\end{enumerate}
Teams that do not satisfy the Eligibility Decision Tree (see appendix \ref{EligibilityDecisionTree}), are usually allowed as spectator if there is room for them.

\section{General Concept of the Contest}
Each team (with a maximum of three students) is trying to solve as many problems (most of the time between the $10$ and the $13$) as possible in five hours, by programming a program on the computer. The program reads the input from the input file, search or computes the right answer and gives back the results as output.

Most of the time, the problems are based on known classic algorithms, like the shortest path algorithm of Dijkstra or backtracking. Often there are a number of mathematical problems.

Logically, may only discuss with their teammates. Furthermore, it is allowed to use a cheat sheet (Team Reference Document) and to bring their own keyboard. Teams that have a problem correctly within four hours, gets a balloon of a balloon babe in the color of the problem.

\section{Sending in a Submissions}
Via the jury interface DomJudge (a website), teams can hand in the submissions. Further access to the internet has been blocked during the contest. The jury interface also tells the teams, if the submissins was correct or not. The program will be rated on the following two criteria: accuracy and efficiency.

Dat wil zeggen: het programma moet een vooraf bepaalde set testcases (dat kan ook een case zijn die niet gegeven is als input in de problems) correct oplossen binnen een vooraf vastgestelde tijdslimiet (meestal enkele seconden).

De mogelijke reacties van de jury zijn onder andere:
\begin{itemize}
\item Accepted;
\item Wrong Answer;
\item Timelimit Exceeded;
\item Runtime Error.
\end{itemize}
Voor elke goede oplossing (Accepted) krijgt men een ballon van een balloon babe.

\section{Programmeertalen}
De onderstaande talen zijn toegestaan bij de genoemde contests:
\begin{itemize}
\item[AAPP] Java, C, C++, C++11, C\#, Haskell
\item[BAPC] Java, C, C++
\item[NWERC] Java, C,  C++
\item[ICPC World Finales] Java, C, C++
\end{itemize}

\section{Puntentelling}
De score van een team bestaat uit twee onderdelen:
\begin{itemize}
\item Het aantal opgaven opgelost in vijf uur
\item De totale tijd (de penalty time)
\end{itemize}
Per opgeloste opgave bestaat dat uit de volgende twee onderdelen:
\begin{itemize}
\item Het aantal minuten tussen het begin van de wedstrijd en het oplossen van de opgave
\item $20$ strafminuten voor elke foute inzending
\end{itemize}
Een foute inzending levert alleen strafminuten op als de opgave later alsnog wordt opgelost.

De teams worden gesorteerd op aantal opgeloste opgaven. Teams met evenveel opgaven worden gesorteerd op tijd. Het laatste uur wordt het scorebord bevroren (freeze). Je hoort dan nog wel of je eigen inzendingen goed of fout zijn, maar niet welke opgaven de andere teams nog oplossen.

Zie voor een uitgebreider uitleg het reglement, Appendix \ref{Rulebook} (Judgement).

\section{Tijdschema}
Elke programmeerwedstrijd heeft dezelfde opzet qua tijdschema. Sommige kiezen er echter voor om deze indeling uit te rekken over een weekend, om zo meer ruimte te cre\"eren voor excursies en sponsoring praatjes. Dit is meestal het geval bij NWERC, omdat deelnemende teams ook uit het buitenland komen. Door teams wordt het uitrekken van de planning vooral als vervelend ervaren, omdat het evenement veel langer duurt dan nodig is.

Het tijdschema ziet er als volgt uit:
\begin{itemize}
\item[Registratie] Het registreren van teamsleden. Organisatie deelt de verplichte gesponserde T-shirts en geeft de eventuele goodiebags. Eventueel controleren en innemen van cheatsheets. Toetsenborden kunnen eventueel ook gecontroleerd worden ($30$ minuten).
\item[Welkomswoord] Teams worden verwelkomt door de voorzitter en de hoofdsponsor. Uitleg van de spelregels, het doornemen van de dagplanning en een praatje van de hoofdsponsor (maximaal $30$ minuten).
\item[Testsessie] Voor de contest is er een testsessie. Deze testsessie wordt gebruikt om te kijken of de wedstrijdomgeving aan alle verwachtingen voldoet (maximaal $1$ uur).
\item[Coach meeting] Meeting voor de coach, zie ook sectie \ref{CoachMeeting} (maximaal $30$ minuten).
\item[Lunch] Meestal verzorgd door de hoofdsponsor ($1$ uur).
\item[Last remarks] Laatste uitleg, vragen of teams nog brandende vragen hebben (hooguit een kwartiertje, maar er wordt $30$ minuten gerekend zodat iedereen tijd heeft om op zijn plek te gaan zitten voor de contest).
\item[Contest] Contest begint ($5$ uur).
\item[Freeze scoreboard] Scoreboard staat op freeze, men kan alleen hun eigen inzendingen zien en of deze goed zijn of niet ($4$ uur na dat de contest is begonnen).
\item[Borrel] Borrelen ($1$ uur)
\item[Prijsuitreiking] Uiteindelijke scoreboard wordt gepresenteerd ($10$ minuten)
\item[Presentatie problems] Presentatie met de oplossingen wordt gepresenteerd door de jury ($15$ minuten).
\item[Dinner] Optioneel: hangt er vanaf of de commissie geld heeft om het eten te vergoeden voor de teams.
\end{itemize}