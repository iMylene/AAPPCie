\section{Preliminaries}
For students of the VU Amsterdam, the International Collegiate Programming Contest (ICPC) knows four rounds:
\begin{enumerate}
\item Amsterdam Algorithm Programming Preliminaries (AAPP): Open for all VU Amsterdam teams which satifsy the conditions, see also the Appendix \ref{EligibilityDecisionTree}. The top three teams are allowed to BAPC.
\item Benelux Algorithm Programming Contest (BAPC): The top three teams of each educational institution can accompete to BAPC. The top two teams of  each educational institution are allowed to NWERC, provided that the team consists three team members (see also Appendix \ref{EligibilityDecisionTree}).
\item Northwestern Europe Regional Contest (NWERC): The top two teams of each educational institution can accompete to NWERC. The top three teams of  each educational institution are allowed to the ICPC World Finales.
\item International Collegiate Programming Contest (ICPC) World Finales: The top three teams of the region Northwestern Europe are allowed to the ICPC World Finals.
\end{enumerate}
Teams that do not satisfy the Eligibility Decision Tree (see appendix \ref{EligibilityDecisionTree}), are usually allowed as spectator if there is room for them.

\section{General Concept of the Contest}
Each team (with a maximum of three students) is trying to solve as many problems (most of the time between the $10$ and the $13$) as possible in five hours, by programming a program on the computer. The program reads the input from the input file, search or computes the right answer and gives back the results as output.

Most of the time, the problems are based on known classic algorithms, like the shortest path algorithm of Dijkstra or backtracking. Often there are a number of mathematical problems.

Logically, may only discuss with their teammates. Furthermore, it is allowed to use a cheat sheet (Team Reference Document) and to bring their own keyboard. Teams that have a problem correctly within four hours, gets a balloon of a balloon babe in the color of the problem.

\section{Sending in a Submissions}
Via the jury interface DomJudge (a website), teams can hand in the submissions. Further access to the internet has been blocked during the contest. The jury interface also tells the teams, if the submissins was correct or not. The program will be rated on the following two criteria: accuracy and efficiency.

That means: the program has solve a set of test cases (not only the one given as input in the problem) in previously set time limit (most of the time a few secondes).

The following reactions of the jury are possible:
\begin{itemize}
\item Accepted;
\item Wrong Answer;
\item Timelimit Exceeded;
\item Runtime Error.
\end{itemize}
For each correct solution (Accepted), you will get a balloon from a balloon babe (if the correct solution was hand in within the first four hours).

\section{Program languages}
The following program languages are accepted at the contest:
\begin{itemize}
\item[AAPP] Java, C, C++, Python
\item[BAPC] Java, C, C++
\item[NWERC] Java, C,  C++
\item[ICPC World Finales] Java, C, C++
\end{itemize}

\section{Score}
The team score depends on two parts:
\begin{itemize}
\item The number of correct solved problems within the five hours
\item The total time (including the penalty time).
\end{itemize}
%
There are two parts per solved problem:
\begin{itemize}
\item The numbers of minutes since the start of the contest and the moment till solving the problem %Het aantal minuten tussen het begin van de wedstrijd en het oplossen van de opgave
\item $20$ minutes penalty time for each wrong solution %strafminuten voor elke foute inzending
\end{itemize}
A wrong solution only gives penalty time if the problem is solved later the contest.

On the scoreboard, teams will be sorted by the numbers of solved problems. Teams who has same amount of solved problems, will be sorted on total time. In the last hour of the contest, the scoreboard will be freezed. You will see from your own team if a problem is correct or not, but it is not possible to see on the scoreboard if others teams have hand in a (correct) solution.

See for more information the rulebook, Appendix \ref{Rulebook} (Judgement).

\section{Time Schedule}
Each program contest has the same set up when its come to time. Some organisations decided to spread this event over a weekend, to create some room for excursions and sponsoring activities. This is the case at NWERC, since teams are coming from foreign countries. However, teams likely do not like it that it takes that long before they can really start with the contest.

The time schedule for the contest is as follow:
\subsection{Registration (total time: $30$ - $40$ minutes)}
Registration of the incoming teams. Showing up as a full team is very practical instead of showing up as an indiviual team member. The organisation handout the following things to the teams (and there coach):
\begin{itemize}
\item Finale remark sheet (advice, hints and general information from the jury);
\item Goodiebags (most of the time including pens, promotion flyers, a contest book and a mug/cup with the organisation logo on it);
\item Sponsored T-shirt, which we are mandatory to wear during the day(s) or contest;
\item Name badges (including teamname).
\end{itemize}
\textbf{Tip: Most of the organisations provide some coffee and tea since the registrations are always early in the morning.}

The teams need to hand in the following things to the organisation:
\begin{itemize}
\item Keyboard (one per team, wireless is not accepted)
\item Cheatsheets (maximum of $3$ per team)
\end{itemize}

Furthermore, the organisation can:
\begin{itemize}
\item Check if there are any teammembers who are allergic for some kind of food and forgot to mention this to the organisation;
\item Control if there is any missing information in ICPC;
\item Take a beautiful team picture (which can be show at the Award Ceremony).
\end{itemize}

\subsection{Opening Ceremony (total time: $30$ minutes)}\label{openingCeremony}
Teams are being welcomed by the chairman or the contest director, and sometimes the main sponsor.

The following things can be mention at the introduction presentation:
\begin{itemize}
\item Statistics about numbers of
	\begin{itemize}
	\item Student;
	\item Teams;
	\item Universities;
	\item Countries
	\end{itemize}
compared to last year;
\item Time schedule of the day/weekend, and if there are any questions about that;
\item That it's mandatory to wear the T-shirt during the contest;
\item The rules of the contest;
\item Introducing the:
	\begin{itemize}
	\item Organisation;
	\item Jury;
	\item Balloon Babes;
	\item Main sponsor
	\end{itemize}
to the teams with names, pictures and there T-shirt color;
\item Welcome speech by the main sponsor.
\end{itemize}

\subsection{System Introduction (total time: $30$ minutes)}
Short presentation about the system, it is more convenient for a Tech member to give this presenation. The following things can be mention at the introduction presentation:
\begin{itemize}
\item What teams can bring on the contest floor;
\item Overview of what teams can use for there submissions, such as
	\begin{itemize}
	\item Accepted language;
	\item Compiler;
	\item Version;
	\item Flags;
	\end{itemize}
\item How the test session will work (and that there in an envelope);
\item Short questions about the system.
\end{itemize}

\subsection{Test session (total time: $60$ minutes)}
Before the contest starts, there is a test session to get used to the system. The problems are in an envelope which can be opened when the organisation said so. This test session is used to see if the contest environment fulfills all the expectations of the organisation and the participants.

\subsection{Coach meeting (total time: $60$ minutes)}
Meeting for the coaches, which can be held during the test session or the contest. See also section \ref{CoachMeeting} for more inside information.

\subsection{Lunch (total time: $60$ minutes)}
Lunch, most of the time catered by the main sponsor.

\subsection{Last remarks (total time: $30$ minutes)}\label{lastRemarks}
The latest issues will be discussed and the last questions of teams will be answered. Takes maximum a quarter of an hour, but we take $30$ minutes into account so that everybody has the time to take there seat at the contest floor before the contest starts.

\subsection{Contest (total time: $300$ minutes)}
Contest starts ($5$ hours), only the participants and the organisation are allowd on the contest floor. Coaches need to wait in the coach room.

\subsection{Freeze scoreboard (total time: $60$ minutes)}
Scoreboard is freezed. Teams can only see there own submission and if they are correct on the scoreboard. The judges can see the realtime scoreboard, while the coaches see the freezed version.

\subsection{Drinks (total time: $60$ minutes)}
Drinks after the contest, to pass time untill the award ceremony starts.

\subsection{Award ceremony (total time: $10$ minutes)}\label{awardCeremony}
Final score board will be present and the prizes will be handed out at the award ceremony.

\subsection{Presentation problems (total time: $15$ minutes)}
Presentation with solutions, which will be presented by the judges.

\subsection{Optional: Dinner}
Depends if the organisation has the money to compensate the cost for the participants.