\label{Functieverdeling}
\section{Chairman}
	Leads the commission and ensures that everything runs smoothly. He delegates and motivates the members of the commission to do there tasks (see also section \ref{motivated}). He also leads the meetings (see also section \ref{leadingMeetings}). Carries responsibility for the committee. \label{responsibilityChairman} Read also the responsibility of a contest director: \ref{responsibilityChairman}.
	
	\subsection{Leading the Meetings}\label{leadingMeetings}
	Plan regular meetings, but only do so when its \underline{necessary} because nobody likes pointless meetings. Make sure everyone gets the chance to bring something up during the meeting.
	
	Design for the agenda is as follow (Dutch/English):
	\begin{enumerate}
	\item Opening
	\item Agenda
	\item Notulen Vorige Vergadering (NVV) / Minutes of Previous Meeting
	\item Post, Email en Mededelingen (PEM) / Post, Receive Email and Announcements
	\item Oude Actiepunten / Old Action Points
	\item \textit{\{kies een onderwerp\}} / \textit{\{choose a subject\}}
	\item \textit{\{kies een onderwerp\}} / \textit{\{choose a subject\}}
	\item Wat Verder Ter Tafel Komt (WVTTK) / Any Other Business (AOB)
	\item Samenwerkingsrondje (censuur) / Informal Round (censorship)
	\item Datum Volgende Vergadering (DVV) / Date Next Meeting
	\item Sluiting / Closing
	\end{enumerate}
	
	And here is a timeline for planning a meeting:\newline
	
	\begin{vtimeline}[timeline color=green!80!blue,description={text width=11cm}, 
	row sep=2ex, 
	use timeline header,
	timeline title={Plan a Meeting 101}]
	Day 0 & Think if you have time to have a meeting \endlr	
	Day 1 & Send scheduler + action points \endlr
	Day 5 & Reminder: fill in the scheduler \endlr
	Day 6 & Read the minutes \endlr
	Day 6 & Set up the agenda \endlr
	Day 7 & Mail: date of the meeting + minutes + agenda + action points \endlr
	Day 7 & Reserveve: conference room \endlr
	Day 8 - 14 & Print: minutes + agenda \endlr
	Day 8 - 14 & Reminder: meeting \endlr
	Day 8 - 14 & Meeting! \endlr
	Sometime & Wait for the rough copy of the minutes \endlr
	Sometime & Do your action points! \endlr
	\end{vtimeline}
	
	\subsubsection{Motivated}\label{motivated}
	Keeping members motivated can ensure that they would like to continue to help the committee. In order to ensure that they remain motivated, there must be a clear goal. Not only the main goal (to organise a beautiful contest for our participants), but also the action points should be as clear as possible otherwise these are not carried out.
	
	Furthermore, its helps reminding members to do there action points and to ask (friendly) if they have finished there action points. It also helps to ask if they need any help from other members to carry out the task.
	
	Solidarity with the committee also ensures that a member gets more sense to carry out their work. Eating together before/after the meeting, pie during the meeting and after the meeting working together at a task in the Stuka works.
	
	\subsection{Presentations and Speeches}
	Somebody needs to talk at
	\begin{itemize}
	\item the opening ceremony (see also \ref{openingCeremony});
	\item the last remarks (see also \ref{lastRemarks});
	\item the award ceremony (see also \ref{awardCeremony});
	\end{itemize}
	and that person is you! Prepare on time your presentations (save as PDF) and speeches.
	
	\subsection{Tips}
	\begin{itemize}
	\item Start on time with organising the contest!
	\item Keep in touch with important people:
		\begin{itemize}
		\item Board members;
		\item Chairmen of other BAPC committee in the Benelux;
		\item Coaches in the Benelux.
		\end{itemize}
	\item Do every week something for your committee.
	\end{itemize}
	
\section{Secretary}
	\subsection{Maintain the Mailbox}
	Label incoming mail and processing:
	\begin{enumerate}
	\item Forward to the right person(s)
	\item Answering
	\item Archive 
	\item Delete	
	\end{enumerate}	
	
\section{Treasurer}
	\subsection{Keep Track of Finances}
	
	\subsection{Budget}
	
	\subsection{Financial Report}
	
	\subsection{Funding}
	
\section{Assessor}
	
\section{Contest Director}
Charman of the contest, decides together with the judges (see also \ref{judge}) and the tech (see also \ref{tech}) if the contest can start and makes decisions about the contest. Carries responsibility for the contest. \label{responsibilityContestDirector} Read also the responsibility of a chairman: \ref{responsibilityChairman}.

\section{Sponsoring\label{sponsoringTasks}}
Commissar extern takes care of sponsoring. Sponsoring is needed to compensate the costs, to have a beautiful location and to give better prizes to the winners. See also \ref{sponsoring} for more inside information.
	
\section{Tech}\label{techTasks}
The tech is taking care of the technical part of organising a programming contest, see also \ref{tech} for more inside information.

\section{Judge}\label{judge}

\section{Balloon babes}
Als teams een opdracht goed hebben binnen vier uur, krijgen ze een ballon van een balloon babe. Deze schoonheid zal deze ballonen netjes vastzetten aan het bureau of aan een bureaustoel. Naast deze zware taak, doen ze ook het volgende voor de wedstrijd:
\begin{itemize}
\item Helpen bij de registratiebalie
\item Controleren dat niemand begint, voordat de contest is begonnen
\item Controleren dat niemand de opgave opent, voordat de contest is begonnen
\item Mobieltjes laten inleveren.
\end{itemize}
{\color{white}test}\\
%%%
En tijdens de wedstrijd voeren ze de volgende taken uit:
\begin{itemize}
\item Ballonnen uitdelen
\item Printjes aangeven
\item Meelopen (roken, wc, eten)
\item Controleren dat men niet vals speelt.
\end{itemize}

\section{Coaches}
Coach zijn van een team bij BAPC of NWERC, houdt in dat je het team ondersteunt daar waar nodig is en om de commissie te vertegenwoordigen bij eventuele coachmeetings 

	\subsection{Voorbereiding}
	Ondersteuning geven aan een team houdt in dat je het onderstaande zal moeten regelen of voorbereiden:
	\begin{itemize}
	\item Vervoer naar de wedstrijd (auto, OV etcetera)
	\item De registratie, zodat men kan deelnemen aan de competitie. Sowieso invoeren in ICPC, maar soms zijn er ook aanvullende formulieren (the local registration form).
	\item Koffie vinden
	\item Aanmelden bij de registratie balie. Vergeet dan niet de onderstaande informatie bij de hand te hebben:
		\begin{itemize}
		\item 	Teamnamen
		\item 	Totaal aantal personen inclusief coach(es), voor het aantal goodiebags
		\item 	Aantal mannen, aantal vrouwen inclusief coach(es), voor het geval dat er wel vrouwen T-shirts zijn.
		\item 	T-shirtmaten (voor vrouwen geldt dat als het unisex/mannen T-shirts zijn, een maat kleiner nemen dan normaal gesproken wordt gekozen)
		\end{itemize}
	\item Cheatsheets uitprinten/regelen
	\item Training regelen of zorgen dat ze zelf trainen
	\item Het team eraan herinneren dat je je eigen toetsenbord mee mag nemen
	\end{itemize}
	\textbf{Tip: het kan handig zijn om een Whatsapp groep te starten met de teamleden en de coaches erin.}

	Nadat de wedstrijd is begonnen, mag de coach zich niet meer op de contest ground bevinden. \textbf{Tip: neem wat mee om de vijf uur door te komen.}
	
	\subsection{Coach meeting}\label{CoachMeeting}
	General meeting:
	\begin{itemize}
	\item Current contest update
	\item Next year contest location (update of the bid $2016$-$2017$, geen slaapplekken in Bath. Goedkoopste optie is YHA Bath. Maximaal $120$ teams, zie ook http://staff.bath.ac.uk/masjhd/NWERC/.)
	\item Future location. Temp. bid. Prio voor landen waar NWERC nog niet is geweest.
	\end{itemize}		
	
	Meestal wordt er tijdens de test sessie of tijdens de wedstrijd, een coach meeting gehouden om eventuele problemen over de wedstrijd zelf te kunnen bespreken. Omdat je de commissie/universiteit vertegenwoordigd, is het handig om met de commissie de onderstaande vragen te bespreken voor vertrek. De volgende vragen kunnen daarnaast nog besproken worden:
	\begin{itemize}
	\item Welke universiteit organiseert de volgende editie(s) van de wedstrijd?
	\item Hoe zijn de voorrondes gegaan?
	\end{itemize}	
