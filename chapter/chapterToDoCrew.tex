\section{Organisatie}\label{Functieverdeling}
	\subsection{Voorzitter}
	De voorzitter leidt de commissie en zorgt ervoor dat alles in goede banen loopt. Hij/zij delegeert en motiveert de leden om hun taken uit te voeren (zie ook sectie \ref{Motiveren}). Tevens leidt hij/zij de vergaderingen (zie ook sectie \ref{Vergaderen}).
	
	\subsubsection{Vergaderen}\label{Vergaderen}
	Vergader regelmatig, maar alleen als het ook nuttig en nodig is voor de commissie. Zorg ervoor dat iedereen de kans krijgt om iets in te brengen tijdens de vergadering.
	
	\begin{vtimeline}[timeline color=green!80!blue,description={text width=11cm}, 
	row sep=2ex, 
	use timeline header,
	timeline title={Vergadering plannen 101}]
	Dag 0 & Bedenken of je volgende week tijd heb om te vergaderen \endlr	
	Dag 1 & Vergaderplanner sturen + actiepunten \endlr
	Dag 5 & Reminder: planner invullen \endlr
	Dag 6 & Notulen doorlezen of aan de notulist vragen of ze af zijn \endlr
	Dag 6 & Agenda opstellen \endlr
	Dag 7 & Mailen: datum vergadering + notulen + agenda + actiepunten\endlr
	Dag 7 & Reserveren: vergaderzaal \endlr
	Dag 8 - 14 & Printen: notulen + agenda \endlr
	Dag 8 - 14 & Reminder: vergadering \endlr
	Dag 8 - 14 & Vergaderen! \endlr
	Ooit & Wachten op de (klad)notulen \endlr
	\end{vtimeline}
		
	Hier is een opzet voor de agenda:
	\begin{enumerate}
	\item Opening
	\item Agenda
	\item Notulen Vorige Vergadering (NVV)
	\item Post, Email en Mededelingen (PEM)
	\item Oude Actiepunten
	\item \textit{\{kies een onderwerp\}}
	\item \textit{\{kies een onderwerp\}}
	\item Wat Verder Ter Tafel Komt (WVTTK)
	\item Samenwerkingsrondje
	\item Datum Volgende Vergadering (DVV)
	\item Sluiting
	\end{enumerate}
		
	\subsubsection{Motiveren}\label{Motiveren}
	Het enthousiast kunnen motiveren en coachen van leden, kan ervoor zorgen dat men graag willen blijven helpen binnen de commissie. Om er voor te zorgen dat men gemotiveerd blijft, moet er een duidelijk doel zijn. Niet alleen het doel (het organiseren van de contest), maar ook de actiepunten moeten duidelijk zijn anders worden deze niet uitgevoerd.
	
	Verder helpt het om leden eens in de zoveel tijd te herinneren aan hun taken of actiepunten, en om (vriendelijk) te vragen of ze deze al hebben uitgevoerd. Het helpt ook om te vragen of ze hulp van andere leden nodig hebben om de taak uit te voeren.
	
	Verbondenheid met de commissie zorgt er ook voor dat een lid meer zin krijgt om hun werk uit te voeren. Samen eten voor de vergadering, taart tijdens de vergadering en na de vergadering samen aan de slag gaan in de Stuka wil wel werken.
	
	\subsection{Secretaris}
	\subsubsection{Mail bijhouden}
	Binnenkomende mail labelen en verwerken:
	\begin{enumerate}
	\item Doorsturen naar de juiste persoon of personen
	\item Beantwoorden
	\item Archiveren 
	\item Verwijderen	
	\end{enumerate}	
	
	\subsection{Penningmeester}
	\subsubsection{Financi\"en bijhouden}
	
	\subsubsection{Begroting}
	
	\subsubsection{Afrekening}
	
	\subsubsection{Subsidie}
	
	\subsubsection{Bijzitter}
	
	\subsection{Contest Director}
	Voorzitter van de contest, bepaald samen met de jury en de tech of de contest kan beginnen en maakt besluiten over de contest. Draagt verantwoordlijkheid voor de contest.

\section{Sponsoring\label{Sponsoring}}
	
\section{Tech}

\section{Jury}

\section{Balloon babes}
Als teams een opdracht goed hebben binnen vier uur, krijgen ze een ballon van een balloon babe. Deze schoonheid zal deze ballonen netjes vastzetten aan het bureau of aan een bureaustoel. Naast deze zware taak, doen ze ook het volgende voor de wedstrijd:
\begin{itemize}
\item Helpen bij de registratiebalie
\item Controleren dat niemand begint, voordat de contest is begonnen
\item Controleren dat niemand de opgave opent, voordat de contest is begonnen
\item Mobieltjes laten inleveren.
\end{itemize}
{\color{white}test}\\
%%%
En tijdens de wedstrijd voeren ze de volgende taken uit:
\begin{itemize}
\item Ballonnen uitdelen
\item Printjes aangeven
\item Meelopen (roken, wc, eten)
\item Controleren dat men niet vals speelt.
\end{itemize}

\section{Coaches}
Coach zijn van een team bij BAPC of NWERC, houdt in dat je het team ondersteunt daar waar nodig is en om de commissie te vertegenwoordigen bij eventuele coachmeetings 

	\subsection{Voorbereiding}
	Ondersteuning geven aan een team houdt in dat je het onderstaande zal moeten regelen of voorbereiden:
	\begin{itemize}
	\item Vervoer naar de wedstrijd (auto, OV etcetera)
	\item De registratie, zodat men kan deelnemen aan de competitie. Sowieso invoeren in ICPC, maar soms zijn er ook aanvullende formulieren (the local registration form).
	\item Koffie vinden
	\item Aanmelden bij de registratie balie. Vergeet dan niet de onderstaande informatie bij de hand te hebben:
		\begin{itemize}
		\item 	Teamnamen
		\item 	Totaal aantal personen inclusief coach(es), voor het aantal goodiebags
		\item 	Aantal mannen, aantal vrouwen inclusief coach(es), voor het geval dat er wel vrouwen T-shirts zijn.
		\item 	T-shirtmaten (voor vrouwen geldt dat als het unisex/mannen T-shirts zijn, een maat kleiner nemen dan normaal gesproken wordt gekozen)
		\end{itemize}
	\item Cheatsheets uitprinten/regelen
	\item Training regelen of zorgen dat ze zelf trainen
	\item Het team eraan herinneren dat je je eigen toetsenbord mee mag nemen
	\end{itemize}
	\textbf{Tip: het kan handig zijn om een Whatsapp groep te starten met de teamleden en de coaches erin.}

	Nadat de wedstrijd is begonnen, mag de coach zich niet meer op de contest ground bevinden. \textbf{Tip: neem wat mee om de vijf uur door te komen.}
	
	\subsection{Coach meeting}\label{CoachMeeting}
	Meestal wordt er tijdens de test sessie of tijdens de wedstrijd, een coach meeting gehouden om eventuele problemen over de wedstrijd zelf te kunnen bespreken. Omdat je de commissie/universiteit vertegenwoordigd, is het handig om met de commissie de onderstaande vragen te bespreken voor vertrek. De volgende vragen kunnen daarnaast nog besproken worden:
	\begin{itemize}
	\item Welke universiteit organiseert de volgende editie(s) van de wedstrijd?
	\item Hoe zijn de voorrondes gegaan?
	\end{itemize}	
